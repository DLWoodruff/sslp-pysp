\documentclass{article}

% Language setting
% Replace `english' with e.g. `spanish' to change the document language
\usepackage[english]{babel}

% Set page size and margins
% Replace `letterpaper' with`a4paper' for UK/EU standard size
\usepackage[letterpaper,top=2cm,bottom=2cm,left=3cm,right=3cm,marginparwidth=1.75cm]{geometry}

% Useful packages
\usepackage{amsmath}
\usepackage{amsthm}
\usepackage{amsfonts}
\usepackage{graphicx}
\usepackage{algorithm2e}
\usepackage{enumitem}
\usepackage{comment}
\usepackage{natbib}

\DeclareMathOperator*{\argmin}{arg\,min}
\newcommand{\xbar}{\overline{x}}

\usepackage[colorlinks=true, allcolors=blue]{hyperref}

\title{Wall Clock Time for the SSLP: 2023 Update}
\author{  David L Woodruff\\
  Graduate School of Management\\
  \\
  University of California Davis\\
  Davis CA 95616 USA}
\date{\today}

\newtheorem{theorem}{Theorem}
\newtheorem{lemma}{Lemma}

\begin{document}
\maketitle

\abstract{Simple experiments using the stochastic server location problem (SSLP).}

\section{Introduction}

As noted in the version of this work done seven years ago \cite{sslpfortime}, the stochastic server location problem (SSLP) \cite{sslp} continues to serve as a classic vehicle
for testing and improving algorithms for optimization under uncertainty. In this note we
examine the issue of finding solutions with reasonably tight bounds with as little elapsed
time as possible using a particular mid-sized workstation. Of course, parallel computing is needed, but apologies must be extended
to readers with an interest in classic research in parallel computing. Our interest here is
in the practical question of wall-clock time on a particular computer and little else.
We use Progressive Hedging (PH) \cite{ph} with lower bounds \cite{gadeetal16} 
and an incombent finder as implemented in the mpi-sppy library \cite{mpi-sppy}.
 The extensions described in \cite{WW} are not used in the interest of keeping it simple.

The SSLP instances have names that include values for m, n, and s in that order. The value
of m gives the number of potential server locations, which n give the number of potential
clients. Together, they determine the size of the problem instance for one scenario. The
value of s gives the number of scenarios sampled for the instance.

\section{Results on a 48 CPU Computer}

All experiments are conducted on a workstation with 96 Intel(R) Xeon(R) E7-4830 v3 processors running at 2.10GHz
and 1TB of RAM (48 physical cores).  This is the same computer that was used seven years ago \cite{sslpfortime}.

\subsection{Gurobi without decomposition}

Tables~\ref{tab:ef} show results for Gurobi 10.0 

\end{table}
\begin{table}
\begin{centering}
\begin{tabular}{lrrr}
\hline\hline
instance & time (sec) &   Obj.  & bound \\
\hline
sslp 10 50 50 & 10 & -369.0 & -370.8 \\
              & 1000 & -369.9 & -370.1 \\
\\
sslp 10 50 2000 & 81 & -173.4 & -388.0 \\
\hline\hline
\end{tabular}
\caption{Results for out-of-the box Gurobi 10.0 applied to the extensive form directly with a 32 thread limit. Times do not include model instantiation, so they are just the solve times output by Gurobi.\label{tab:ef}}
\end{centering}
\end{table}


\section{Conclusions and Directions}

Because mpi-sppy scales so well, we might try it on a big computer.

\bibliographystyle{plain}
\bibliography{references}

\end{document}


